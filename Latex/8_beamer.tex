
\documentclass[dvipsnames, aspectratio=169]{beamer}
\usepackage{graphicx}
\usepackage{color}
\title[Tiger]{The Bengal Tiger}
\author[R.R.P.]{Rakesh}
\date{\today}
\institute[IIT Madras]{\large Indian Institute of Technology Madras}
\logo{\includegraphics[height=1cm]{tiger2.png}}

%\usetheme{AnnArbor}
%\usecolortheme{crane}

%\usetheme{Dresden}
%\usecolortheme{beaver}

% \usetheme{Berkeley}
% \usecolortheme{default}

\usetheme{Madrid}
\usecolortheme{default}

% Beamer Customize Themes 
\setbeamercolor{structure}{fg=brown}
% \setbeamercolor{structure}{fg=green}
% \setbeamercolor{structure}{fg=violet}

% Changing the theme color of the pelettes.  
\setbeamercolor{palette tertiary}{bg=yellow, fg=black}    % Changing the theme color of tertiary pelette.
\setbeamercolor{palette secondary}{bg=white, fg=red}      % Changing the theme color of secondary pelette.
\setbeamercolor{palette primary}{bg=blue, fg=white}       % Changing the bg/fg color of primary pelette.

\setbeamercolor{background canvas}{bg=yellow!16}          % Changing the background canvas color for entire ppt.


\begin{document}

  \begin{frame}
    \maketitle
  \end{frame}

  \begin{frame}{Outline}
    \tableofcontents
  \end{frame}
  
  \section{Welcome}
      \begin{frame}[c]
        Welcome everyone !
      \end{frame}
  
  \section{Introduction}
      \begin{frame}[c]{Introduction}
        The Bengal tiger or Royal Bengal tiger is a population of the Panthera tigris tigris \
        subspecies and the nominate tiger subspecies. It ranks among the biggest wild cats \
        alive today. It is estimated to have been present in the Indian subcontinent since \
        the Late Pleistocene for about 12,000 to 16,500 years. Its historical range covered \
        the Indus River valley until the early 19th century, almost all of India, western Pakistan,
        southern Nepal, Bangladesh, Bhutan and southwestern China. Today, it inhabits India,
        Bangladesh, Nepal, Bhutan, and southwestern China.
        It is threatened by poaching, habitat loss and habitat fragmentation!
      \end{frame}

  \section{Images}
      \begin{frame}[c]
        \includegraphics[scale=0.22]{tiger.jpg}
      \end{frame}
      \begin{frame}[c]
        \includegraphics[scale=0.28]{tiger2.png}
      \end{frame}
      \begin{frame}[c]
        \includegraphics[scale=0.28]{tiger2.png}
      \end{frame}

  \section{Blocks}
      \begin{frame}[c]{Basic Blocks}
            \begin{block}{Information}
              This is topic 6
            \end{block}
            
            \begin{alertblock}{Information}
              This is topic 7
            \end{alertblock}
        
            \begin{example}
              $$ 2 + 2 = 4 $$
            \end{example}
      \end{frame}

  \section{Columns}
      \begin{frame}[c]{Basic Blocks}
          \begin{columns}
              \column{0.3\textwidth}
              \begin{block}{Information}
                This is topic 6
              \end{block}
            
              \column{0.3\textwidth}
              \begin{alertblock}{Information}
                This is topic 7
              \end{alertblock}
            
              \column{0.3\textwidth}
              \begin{example}
                $ 2 + 2 = 4 $
              \end{example}
          \end{columns}
      \end{frame}

  \section{Animation}
      \begin{frame}
        \begin{enumerate}
          \item red   \pause
          \item blue  \pause
          \item green \pause
        \end{enumerate}
      \end{frame}

      \begin{frame}
        \begin{enumerate}
          \item<6-> lion   
          \item<4-> deer  
          \item<2-> crow 
          \item<5-> pigeon 
          \item<3-> dog 
          \item<1-> cat 
        \end{enumerate}
      \end{frame}

      \begin{frame}
        \begin{enumerate}
          \item<6-> Rose   
          \item<4-> Carnation  
          \item<2-> Jasmine 
          \item<5-> Lily
          \item<3>  Sunflower
          \item<1-> Lavender
        \end{enumerate} 
      \end{frame}

      \begin{frame}
          \centering
          \begin{table}
            \begin{tabular}{ll}
              \hline
                \onslide<1-> No. & Item \\
              \hline
                \onslide<5-> 1   & Pen    \\
                \onslide<4-> 2   & Pencil \\ 
                \onslide<3-> 3   & Scale  \\
                \onslide<2-> 4   & Eraser \\
              \hline
            \end{tabular}
            \caption{stationary Items}
          \end{table}
      \end{frame}

\end{document}