\documentclass{article}               %# Specifies type of the document
\usepackage{multicol}
\usepackage{lipsum}
\usepackage{color}                    % For basic colors
\usepackage[dvipsnames]{xcolor}       % For more colors
\usepackage{ragged2e}                 % For 'justified' text
\usepackage{setspace}  
\usepackage{enumerate}
\usepackage{amssymb}
\usepackage{mathtools}                % 'amsmath' is part of 'mathtools'
\usepackage{arydshln}                 % 'amsmath' is part of 'mathtools'

\title{Creating first document in \LaTeX}   % # Title of the document
\author{Rakesh}                             % # Author of the document
\date{\today}        

\begin{document}
  \maketitle
  \section{Introduction to Text Formatting}

  
  \textbf{Text formatting} refers to the \emph{process of altering the appearance and layout of text in a 
  document or on a webpage}. This involves applying various \\
  \underline{styles},                         \\
  \textsc{ sizes, colors, and other visual }  \\
  \textrm{enhancements to improve }     \\
  \textsf{readability and aesthetics}.  \\
  \texttt{ This is monospaced text.}    \\
  x\textsuperscript{2}y is a quadratic equation.    \\
  H\textsubscript{2}O is chemical formula of water. \\
  other visual enhancements to improve readability and aesthetics. \\

  \section{Text Fonts}
  
      \tiny          This is tiny text.

      \scriptsize    This is the script sized text.

      \footnotesize  This is footnote sized text.

      \small       This is small sized text.

      \normalsize  This is normal sized text.

      {\large   This is large sized text. }

      {\Large    This is bit large sized text.}

      {\LARGE    This is bit more large sized text.}

      {\huge     This is text with huge size.}

      {\Huge     This is the biggest sized text.}




  \section{Colored Text}

      \pagecolor{Goldenrod!40}
      \textcolor{red}{Red Colored Text}                     \\    
      \textcolor{Plum}{Plum Colored Text}                   \\    
      \textcolor{blue}{Blue Colored Text}                   \\    
      \textcolor{red!70}{70\% red and 30\% white Text}      \\  % 70% red and 30% whilte
      \textcolor{red!70!blue}{Combination of Red and Blue}  \\  % Mixture of red and blue

  \newpage
  \section{Text Alignment }

      \subsection{One : Left Aligned}
      \begin{flushleft}
        This is left aligned text. 
        \lipsum[1]
      \end{flushleft}

      \subsection{Two : Center Aligned}
      \begin{center}
        \lipsum[1]
      \end{center}

      \subsection{Three : Right Aligned}
      \begin{flushright}
        This is Right aligned text. This is Right aligned text. 
        \lipsum[1]
      \end{flushright}

      \subsection{Four : Justified Aligned}
      \begin{justify}
        This is Right aligned text. This is Right aligned text. 
        \lipsum[1]
      \end{justify}


  \newpage
  \section{Vertical and Horizontal Spacing }
      \begin{singlespace}
        \lipsum[1]
      \end{singlespace}


      \begin{onehalfspace}
        \lipsum[1]
      \end{onehalfspace}


      \begin{doublespace}
        \lipsum[1]
      \end{doublespace}

  \newpage    
  \section{Vertical  Spacing }
      
      \paragraph{}
      This is some random text. This is some random text. This is some random text. 
      This is some random text. This is some random text. This is some random text. 
      This is some random text. This is some random text. This is some random text. 
      This is some random text. This is some random text. This is some random text. 
      \smallskip 
      
      This is some random text. This is some random text. This is some random text. 
      This is some random text. This is some random text. This is some random text. 
      This is some random text. This is some random text. This is some random text. 
      This is some random text. This is some random text. This is some random text. 
      \medskip 

      This is some random text. This is some random text. This is some random text. 
      This is some random text. This is some random text. This is some random text. 
      This is some random text. This is some random text. This is some random text. 
      This is some random text. This is some random text. This is some random text. 
      \bigskip
      
      This is some random text. This is some random text. This is some random text. 
      This is some random text. This is some random text. This is some random text. 
      This is some random text. This is some random text. This is some random text. 
      This is some random text. This is some random text. This is some random text. 

      \vspace{20pt}
      This is some random text. This is some random text. This is some random text. 
      This is some random text. This is some random text. This is some random text. 
      This is some random text. This is some random text. This is some random text. 
      This is some random text. This is some random text. This is some random text. 

      \pagebreak
      
      This is some random text. This is some random text. This is some random text. 
      This is some random text. This is some random text. This is some random text. 
      This is some random text. This is some random text. This is some random text. 
      This is some random text. This is some random text. This is some random text. 

      \vfill 
      This is some random text. This is some random text. This is some random text. 
      This is some random text. This is some random text. This is some random text. 
      This is some random text. This is some random text. This is some random text. 
      This is some random text. This is some random text. This is some random text. 
      
      \vfill 
      This is some random text. This is some random text. This is some random text. 
      This is some random text. This is some random text. This is some random text. 
      This is some random text. This is some random text. This is some random text. 
      This is some random text. This is some random text. This is some random text. 

  \pagebreak	
  \section{Horizontal Spacing }

      This is '\ ' single space. \\
      This is '\, ' more than single space. \\
      This is '\hspace{20pt}' space of 20 points. \\
      This occupies \hfill the \hfill entire line. \\ 


  \pagebreak
  \section{Lists }

    Bulleted list
      \begin{itemize}
        \item Blue
            \begin{itemize}
              \item Dark Blue
              \item Light Blue
            \end{itemize}
        \item Red
            \begin{enumerate}[i.]
              \item Dark Red
              \item Light Red
            \end{enumerate}
        \item Green
            \begin{enumerate}[(a)]
              \item Teal Green 
              \item Dark Green
            \end{enumerate}
      \end{itemize}


  \noindent
  \subsection{Fruits}
      \begin{itemize}
        \renewcommand{\labelitemi}{$\diamond$} 
        \renewcommand{\labelitemii}{$\blacksquare$} 
        \item Citrius Fruits
              \begin{itemize}
                \item Lemon 
                \item Lime 
                \item Oranges
              \end{itemize}
        \item Tropical Fruits
              \begin{itemize}
                \item Mango
                \item Papaya
              \end{itemize}
      \end{itemize}


      \noindent
  \subsection{Numbered List}
          \begin{enumerate}
            \item One
            \item Two
            \item Three
          \end{enumerate}
        
      \noindent
      
  \subsection{Description List}
        \begin{description}
          \item [Input Devices] : mouse, keyboard
          \item [Output Devices] : Monitor, Printer
        \end{description}

  \noindent
  \section{Math Expressions}
      \subsection{Inline Equations}
        $2^{2} + 2^{2} = 8$   \\
        $\sqrt[4]{4096} = 8 $ \\
        The union of two sets A and B is denoted as 
        $ A \cup B = \{ x \in A \  \text{or} \  x \in B \} $

       \subsection{ Blocked Equations }
        $$2 \times 2 = 4$$
        \[ \sin^2 \theta + \cos^2 \theta = 1  \]  

       \subsection{ These are some rational numbers} 
        $\frac{a}{b}$ \\
        $\frac{a}{\frac{b}{c}} \ge 1 $
        
       \subsection{ These are some Additional rational numbers }
        $$ \frac{a}{b} $$ \\
        \[ \frac{a}{\frac{b}{c}} \ge 1 \]

        $$ \Bigg[ \Big \{  \frac{a}{b}    \Big\}    \times
                  \Big \{  \frac{c}{d}    \Big\}
            \Bigg]  $$

       \subsection{ Summation }
       
        $$ \sum_{i=1}^{b} g(i)=0  \text{ for } b<a $$
        \[ \sum_{i=1}^{n} i = \frac{n(n+1)}{2}   \] 
        
       \subsection{ Integration}
        $$ \int_{min}^{max} $$                  

        % Integration
        $$ \int_{0}^{\infty} f(x)dx $$
       
       \subsection{ Limits }
       $$ \lim_{x \to c} f(x) = L $$

        
  \noindent
  \section{Matrix}
        
    \subsection{p matrix}
        $$
        \begin{pmatrix}
        1 & 2 \\
        3 & 4 \\
        \end{pmatrix}
        +
        \begin{pmatrix}
          5 & 6 \\
          7 & 8 \\
          \end{pmatrix}
        = 
        \begin{pmatrix}
          6 & 8 \\
          10 & 12 \\
          \end{pmatrix}
        $$
        
	\subsection{b matrix}
	$$
	\begin{bmatrix}
		1 & 2 \\
		3 & 4 \\
	\end{bmatrix}
	+
	\begin{bmatrix}
		5 & 6 \\
		7 & 8 \\
	\end{bmatrix}
	= 
	\begin{bmatrix}
		6 & 8 \\
		10 & 12 \\
	\end{bmatrix}
	$$

        
	\subsection{Bmatrix}
	$$
	\begin{Bmatrix}
		1 & 2 \\
		3 & 4 \\
	\end{Bmatrix}
	+
	\begin{Bmatrix}
		5 & 6 \\
		7 & 8 \\
	\end{Bmatrix}
	= 
	\begin{Bmatrix}
		6 & 8 \\
		10 & 12 \\
	\end{Bmatrix}
	$$

        
	\subsection{vmatrix}
	$$
	\begin{vmatrix}
		1 & 2 \\
		3 & 4 \\
	\end{vmatrix}
	+
	\begin{vmatrix}
		5 & 6 \\
		7 & 8 \\
	\end{vmatrix}
	= 
	\begin{vmatrix}
		6 & 8 \\
		10 & 12 \\
	\end{vmatrix}
	$$
        
    \subsection{Vmatrix}
        $$
        \begin{Vmatrix}
        1 & 2 \\
        3 & 4 \\
        \end{Vmatrix}
        +
        \begin{Vmatrix}
          5 & 6 \\
          7 & 8 \\
        \end{Vmatrix}
        = 
        \begin{Vmatrix}
          6 & 8 \\
          10 & 12 \\
        \end{Vmatrix}
        $$

                
  \noindent
  \section{Equations}

     \subsection{Writting numbered Equations} 
        \begin{equation}
          2x + 3y = 89
        \end{equation}

        \begin{equation}
          5x + 8y = 20
        \end{equation}

        \begin{equation}
          x^2 - y^2 = (x+y)(x-y)
        \end{equation}
        
	\subsection{Writting numbered Equations}
        \begin{align}         % Equations are aligned on = sign
          3x - 6 &= 9  \\
          3x &= 9 + 6 \nonumber \\
          3x &= 15  \nonumber \\
          x  &= 5 \nonumber
        \end{align}   





\end{document}