%!TEX program = lualatex
\documentclass{article}
\usepackage{graphicx}
\usepackage{lipsum}
\usepackage{url}
\usepackage{hyperref}
\usepackage{kantlipsum}
\usepackage{imakeidx}

\makeindex[columns=1, title=Back Index, intoc]           

% Set default font to sans serif
% \renewcommand{\familydefault}{\sfdefault}

\usepackage{helvet} % Use Helvetica
\renewcommand{\rmdefault}{phv} % Set sans serif to Helvetica

% \usepackage{fontspec}
% \setmainfont{Arial} % Set main font to Arial

\begin{document}

\begin{titlepage}
\begin{center}
  \Huge
  \textbf{Title of the Project}
  \vfill 
  \LARGE
  \textbf{Submitted by} \\
  \medskip
  Author 1
  \vfill 
  Under the guidance of \\
  \vfill 
  Prof. XYZ Name
  \vfill 
  \includegraphics[width=1in]{iit-madras.png} 
  \vfill
  Department Name
  \vfill
  Institute Name
  \vfill
  Year
\end{center}
\end{titlepage}
  
  \section{Introduction}
  \kant[4]
  
  \section{Hyperlinks and Urls}
  \url{https://www.google.com/} is one of the best search engines. \\ 
  I use \href{https://www.overleaf.com/}{Overleaf} for online editing of the 
  \LaTeX documents. Also, you can refer the subsection \ref{subsec:Requirements} for 
  further details. 

  \section{Background}
  \label{sec:Background}
  The rapid advancement of technology has transformed various sectors, with education being one 
  of the most significantly affected areas. Online learning, once considered a supplementary 
  educational tool, has now become a primary mode of instruction for many institutions worldwide, 
  especially in the wake of the COVID-19 pandemic. \index{pandemic} This shift has prompted 
  researchers to investigate how this mode of learning influences student performance and engagement.
  Prior to the widespread adoption of online learning, traditional classroom settings dominated
  education. However, studies have shown that online learning environments can offer flexibility
  and accessibility that traditional methods may lack (Moore \& Kearsley, 2012). The ability to 
  access course materials from anywhere at any time has made education more inclusive for 
  students with varying needs and schedules. Despite these advantages, concerns have been 
  raised regarding the effectiveness of online learning compared \index{pandemic!learning} to 
  face-to-face instruction.
  Research indicates mixed outcomes regarding student performance in online learning environments.
  Some studies suggest that students in online courses perform equally well or better than their
  peers in traditional settings (Bernard et al., 2004), while others highlight challenges such as
  reduced motivation and higher dropout rates (Lee \& Choi, 2011). These conflicting findings 
  underscore the necessity for further investigation into the factors that contribute to student
  success in online learning.
  This study aims to explore the impact of online learning on student performance by examining key
  variables such as engagement, motivation, and academic achievement. Specifically, it seeks to 
  identify which aspects of online learning environment \index{learning environments} facilitate
  or hinder student success. By addressing these questions, this research will contribute to a 
  deeper understanding of how educational institutions can optimize online learning experiences 
  to enhance student outcomes. In conclusion, as online learning \index{pandemic|see{learning environments}} continues to evolve and expand 
  its reach, it is crucial to 
  assess its implications for student performance. This study will not only fill existing gaps
  in the literature but also provide actionable insights for educators and policymakers aiming 
  to improve online education \index{pandemic!Online Education} frameworks.
  Key Elements Explained
  
  \section{Specification}
  \label{sec:Specification}
  As per the section \ref{sec:Background}, \nameref{sec:Background} from page number \pageref{sec:Background}, we have \kant[4]
  
  \subsection{Requirements}
  It is given in the section \ref{sec:Specification} \label{subsec:Requirements}
  \kant[4]

  \subsection{Experimental setup}
  This section contains experimental setup \cite{ullman} 

  \printindex

  \bibliographystyle{plain}           % Styles - plain, alpha, unsrt, acm 
  \bibliography{7_biblio.bib}             % Specify the bib file in .tex file 

\end{document}